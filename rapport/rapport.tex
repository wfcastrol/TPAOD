\documentclass[a4paper, 10pt, french]{article}
% Préambule; packages qui peuvent être utiles
   \RequirePackage[T1]{fontenc}        % Ce package pourrit les pdf...
   \RequirePackage{babel,indentfirst}  % Pour les césures correctes,
                                       % et pour indenter au début de chaque paragraphe
   \RequirePackage[utf8]{inputenc}   % Pour pouvoir utiliser directement les accents
                                     % et autres caractères français
   \RequirePackage{lmodern,tgpagella} % Police de caractères
   \textwidth 17cm \textheight 25cm \oddsidemargin -0.24cm % Définition taille de la page
   \evensidemargin -1.24cm \topskip 0cm \headheight -1.5cm % Définition des marges
   \RequirePackage{latexsym}                  % Symboles
   \RequirePackage{amsmath}                   % Symboles mathématiques
   \RequirePackage{tikz}   % Pour faire des schémas
   \RequirePackage{graphicx} % Pour inclure des images
   \RequirePackage{listings} % pour mettre des listings
% Fin Préambule; package qui peuvent être utiles

\title{Rapport de TP 4MMAOD : Génération de patch optimal}
\author{
CASTRO LOPEZ Walter Ferney$_1$ (groupe 7$_1$) 
\\ BUALO GIORDANO Lucas Santiago$_2$ (groupe 7$_2$) 
}

\begin{document}

\maketitle
%%%%%%%%%%%%%%%%%%%%%%%%%%%%%%%%%%%%%%%%%%%%%%
\section{Principe de notre  programme}
Nous avons implémenté notre programme avec le méthode récursive. Pour cela, nous avons créé un tableau de dimension $n X m$ où $n$ est le nombre de lignes du fichier original (d'entré) et $m$ est le nombre de lignes du fichier resultat (de sorti); et dans lequelle nous gardons les valeurs des coûts optimals pour calculer les patchs qui transforment le fichiers $A_i$ en $B_j$, cet-à-dire, le fichier qui va la ligne $1$ jusqu'à la ligne $i$ du fichier d'entré, et le fichier qui va de la ligne $1$ jusqu'à la ligne $j$ du fichier de sorti respectivement.

Autre que le coût, on mémorise pour chaque patch optimal la dernière instruction qui a été éfectué, permettant de retrouver le patch optimal une fois on a déjà le coût du patch optimal (le tableau de coût rempli).

Finalement, pour optimiser le temps et limiter les entrées/sorties vers les fichiers, nous gardons dans notre code une copie de chaque fichiers au début, faisant plus facile l'accès à l'information de chaque fichier.

%%%%%%%%%%%%%%%%%%%%%%%%%%%%%%%%%%%%%%%%%%%%%%
\section{Analyse du coût théorique}
    \paragraph{Justification\,: }
Nous allons calculer le coût de chaque fonction dans notre code, tenant en compte que il y en a quelq'unes qui s'appellent entre elles. Le patch comprend une première partie dans laquelle tous les variables grobales qui ont besoin sont initialisés. Et après avec l'appele de la fonction le chemin de coût minimum est calculé. On ne decrira pas les instruction dont le cout n'est pas significative ou celles avec un cout fixe.

  \subsection{Nombre  d'opérations en pire cas\,: }
    \paragraph{Justification\,: }


\begin{itemize}
	\item Décompte de lignes : n $+$ m 
	
	Allocation des arrays de 1 seule dimension qui represent des matrix
 	\item F1 \-- Pour les lignes du “file source” (textes)
 	\item F2 \-- Pour les lignes du “file target” (textes)
 	\item instruction \-- Pour les lignes du differents operation (textes)
	\item optimal \-- Pour les lignes du differents coûts de operation (valeurs)
	\item patch \-- Pour les lignes du chemin optimal (textes)
	
	Asignation des contenues
	\item F1 : $n_1$ (les lignes sont copiés du fichier source au tableau)
	\item F2 : $n_2$ (les lignes sont copiés du fichier target au tableau)
	\item optimal : $(n_1+1)*(n_2+1)$ (les couts sont initaliser en infinito)
	
	Pire Cas de la fonction avec $i=n_1$ et $j=n_2$
	\item DelMul – Allocation d'un tableau pour stocker et analiser après le cas destruction multiple
	\item Un cycle for de taille $n_1$ pour charger le tableau DelMax
	\item Deux appele de la fonction findPatch de manière recursive, qui aura un cout de $(n_1+1)*(n_2+1)$
	\item Generation d'un tableau pour stocker les strings des instructions du chemin optimal : $n_1+n_2$

\end{itemize}

	Si on analyse le cout d'obtenir le chemin optimal, on observe qu'il dèpend surtout des appels recursives, c'est-à-dire que le cout théorique de l'algorythme est d'ordre $O(n_1*n_2)$.
    	
  \subsection{Place mémoire requise\,: }
    \paragraph{Justification\,: }

On va exprimer une approximation de la place en memoire requise, en regardant que un char et un int, les deux occupent un octet :

	\begin{itemize}
	\item F1 : $n_1*$ LINE\_ TAILLE\_ MAX octets.
	\item F2 : $n_2*$ LINE\_ TAILLE\_ MAX octets.
	\item optimal : $(n_1+1)*(n_2+1)$ octets. 
	\item instruction : $(n_1+1)*(n_2+1)*$LINE\_ MAX\_ SIZE octets.
	\item patch : $(n_1+n_2)*$LINE\_ MAX\_ SIZE octets.	
	\end{itemize}

Mais après les differentes appel de malloc pour obtenir les tableaux DelMul ce fait chaque fois jusqu'à l'index de l'appel, cela ajoute un cout de memoire de $n_1!$ octets.

  \subsection{Nombre de défauts de cache sur le modèle CO\,: }
    \paragraph{Justification\,: }
    Nous vu que les défauts de cache. Nous croyons que cela est parce que nous utilisons le méthode récursive et cela posse un problème d'allocation dans la mémoire cache. Chaque fois que le code a besoin d'un tableau, il faut le charger dans la mémoire caché, et comme la mémoire caché n'est pas très grande, il faut l'actualiser plusieur fois.
    
    Si l'on voit le cas del'allocation des fichiers dans ses tableaux, on voit la cause la plus grande des défauts de cache. Tout d'abord, l'ordinateur utilisé pour les test a une mémoire cache de 4 Mo. Nous suposons que pour la mémoire, le nombre de lignes est le même que la longeur de la ligne dedans. Dans ce cas, le nombre de lignes serait environ 1700 et la taille de chaque ligne serait d'environ 1700 octets. De plus, les lignes bien du fichier d'entré ou bien du fichier de sorti fait au maximum 100 octets selon notre modèle. En fin de compte, on utilise une ligne de 1700 octets juste pour alloquer une ligne de 100 octets, ce qui n'est pas du tout optimal, génerant plus des défauts de cache.
    
%%%%%%%%%%%%%%%%%%%%%%%%%%%%%%%%%%%%%%%%%%%%%%
\section{Compte rendu d'expérimentation}
  \subsection{Conditions expérimentaless}

    \subsubsection{Description synthétique de la machine\,:} 
Pour les test, on a laisé l'ordinateur sans autres exécutions pour réduire le temps d'exécution. L'ordinateur à les spécifications suivantes:\\
\\
OS : Ubuntu 14.04.2 LTS \\
CPU : Intel(R) Core(TM) i7-3537U CPU @ 2.00GHz\\
GPU : NVIDIA GF117M [GeForce 610M/710M/820M / GT 620M/625M/630M/720M]\\
RAM : 5855 MiB\\
CACHE: 4096 KB\\
\\
Information trouvé avec les commandes: 

cat /proc/cpuinfo

lshw

lsb\_ release \-- a

    \subsubsection{Méthode utilisée pour les mesures de temps\,: } 
      Pour le temps, nous avons utilisé l'outil \textit{temps} d'ubunu, en disant:

  \subsection{Mesures expérimentales}

	Il faut dire que nous n'avons pas pu faire l'exécution du benchmark 6, parce qu'il y avait un error pour la taille du matrice DelMax, génerant une exeption dans l'éxecution.

    \begin{figure}[h]
      \begin{center}
        \begin{tabular}{|l||r||r|r|r||}
          \hline
          \hline
            & coût         & temps     & temps   & temps \\
            & du patch     & min       & max     & moyen \\
          \hline
          \hline
            benchmark1 & 2486 & 5.989 & 7.342 & 6.634s \\
          \hline
            benchmark2 & 3093 & 58.572s & 1m7.256s & 1m1.299s \\
          \hline
            benchmark3 & 786 & 3m0.393s & 3m16.467s & 3m6.636 \\
          \hline
            benchmark4 & 1638 & 6m.52.723s & 7m39.563s & 7m24.734s \\
          \hline
            benchmark5 & 7376 & 17m28.635s & 18m18.974s & 17m51.708s \\
          \hline
          \hline
        \end{tabular}
        \caption{Mesures des temps minimum, maximum et moyen de 5 exécutions pour les 5 benchmarks.}
        \label{table-temps}
      \end{center}
    \end{figure}

\subsection{Analyse des résultats expérimentaux}
À la fin, les résultats sont un peut trop grandes, tenant en compte le nombre de défauts de cache que l'on fait. C'était impossible de trouver un moyen de comparer expérimentalement la théorie avec la pratique, parce que la matrice DelMul fait une appel a malloc, et donc le temps d'exécution n'ont pas permis réaliser la preuve.

%%%%%%%%%%%%%%%%%%%%%%%%%%%%%%%%%%%%%%%%%%%%%%
\section{Question\,: et  si le coût d'un patch était sa taille en octets ?}
Pour ce cas, il faudrait changer les coûts de chaque opération, et les relattioner avec les octets du patch :

	En cas d'une addition, la première ligne qui a un char avec le “+” et le numéro de la ligne. Et la deuxième ligne s'agit de la ligne à ajoute, ça veut dire (2 + la quantité de chiffre du nombre de caractères à ajouter). Son coût est le nombre de caractères de cette ligne.
	
	Le cas de la substitution c'est similaire.
	
	Pour une supression multiple le coût est similaire me il faut tenir en compte les chiffres des numéros de la ligne d'origin et de la quantité à supprimer.

	En cas d'une supression simple, il y a un coût de 2 + les chiffres du numéro de la ligne.

	Finalement, la copie n'a pas de coût parce qu'il n'ajoute pas des lignes au patch.

\end{document}
%% Fin mise au format

